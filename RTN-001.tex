\documentclass[OPS,authoryear,lsstdraft]{lsstdoc}
% lsstdoc documentation: https://lsst-texmf.lsst.io/lsstdoc.html
\input{meta}

% Package imports go here.

% Local commands go here.

%If you want glossaries
%\input{aglossary.tex}
%\makeglossaries

\title{Data Preview 0: Definition and planning.}


\author{%
William O'Mullane
}

\setDocRef{RTN-001}
\setDocUpstreamLocation{\url{https://github.com/rubin-observatory/rtn-001}}

\date{\vcsDate}

% Optional: name of the document's curator
% \setDocCurator{The Curator of this Document}

\setDocAbstract{%
This document is intended to form the basis for planning DP0 in the interim data facility.
What is DP0 anyway ? What data will be there ?  How will it be served ?
What do we need to have ready when ?
All shall be answered within.
}

% Change history defined here.
% Order: oldest first.
% Fields: VERSION, DATE, DESCRIPTION, OWNER NAME.
% See LPM-51 for version number policy.
\setDocChangeRecord{%
  \addtohist{1}{YYYY-MM-DD}{Unreleased.}{William O'Mullane}
}


\begin{document}

% Create the title page.
%\maketitle
% Frequently for a technote we do not want a title page  uncomment this to remove the title page and changelog.

\mkshorttitle

\section{Introduction}
\tabref{tab:miles} shows the FY21 milestones for the \VRO, many of which
concern, or relate to, data previews.
\secref{sec:dp0} defines what Data Preview 0 is about and covers possible risks and mitigations to that
definition.
\secref{sec:plan} Sets out the planning for achieving DP0.

\tiny \begin{longtable} {|p{0.35\textwidth}|l|l|l|l|l|} \caption{Milestones for Rubin Observatory Data Production and System Perfomrance  FY21 \label{tab:miles}}\\ 
\hline 
\textbf{Milestone}&\textbf{Rubin ID}&\textbf{Year}&\textbf{Q}&\textbf{Type}&\textbf{Team} \\ \hline
{Read only Gen3 butler for DP0 at IDF}&{DP-MW-M03}&{FY21}&{Q1}&{Code Release}&{Science Users Middleware} \\ \hline
{IDF DP0-Ready: Complete IDF installation and IDF staff preparations for DP0.}&{DP-IN-M01}&{FY21}&{Q1}&{Event}&{Data Production} \\ \hline
{Evaluate Batch Production System for DP0.2}&{DP-MW-M07}&{FY21}&{Q1}&{Decision}&{Data Production} \\ \hline
{Develop a model for user support during pre-operations and operations}&{SP-CE-M01}&{FY21}&{Q1}&{Process Definition}&{System Performance} \\ \hline
{DP0.1 Early Access: Provide access to processed images and visit level catalogs from the IDF}&{DP-SR-M02}&{FY21}&{Q2}&{Data Release}&{Data Production} \\ \hline
{HTCondor based worklow system in place}&{DP-MW-M04}&{FY21}&{Q1}&{Code Release}&{Data Production} \\ \hline
{HTCondor based worklow system with tooling (e.g. restart) added.}&{DP-MW-M05}&{FY21}&{Q2}&{Code Release}&{Data Production} \\ \hline
{Gen3 butler and pipeline task ready for production use.}&{DP-MW-M06}&{FY21}&{Q2}&{Code Release}&{Data Production} \\ \hline
{DP0.2 Reprocessing Start: Begin early DRP-like re-processing of DP0 simulated image data, at the IDF.}&{DP-EX-M01}&{FY21}&{Q3}&{Event}&{Data Production} \\ \hline
{Engage with the community to support shared-risk simulated data distribution to community for science with DP0}&{SP-CE-M03}&{FY21}&{Q3}&{Event}&{System Performance} \\ \hline
{Demonstrate EPO interface with DP0}&{DP-SR-M03}&{FY21}&{Q3}&{Process Definition}&{Data Production} \\ \hline
{Deliver beta LSST Data Products Documentation (DP0)}&{SP-CE-M02}&{FY21}&{Q3}&{Code Release}&{System Performance} \\ \hline
{DP0.1 Data Release: science-ready catalogs released from the IDF}&{SP-VV-M01}&{FY21}&{Q3}&{Data Release}&{System Performance} \\ \hline
{USDF Transition Plan: work with selected USDF team to plan start-up of USDF.}&{DP-DM-M05}&{FY21}&{Q4}&{Process Definition}&{Data Production} \\ \hline
{DP0.2 Early Access: Provide access to reprocessed images and visit level catalogs from the IDF}&{DP-SR-M04}&{FY21}&{Q4}&{Data Release}&{Data Production} \\ \hline
{Deploy early instantiation of service desk providing second-tier technical support for community}&{DP-SR-M05}&{FY21}&{Q4}&{Event}&{Data Production} \\ \hline
\end{longtable} \normalsize






\section{Data Preview 0}\label{sec:dp0}
The first ideas about initial releases of \RO data were presented in \citeds{LSO-011}.
There have since been delays in construction such that we have considered making Data Previews with
simulated data or other non \RO data (see \secref{datasets}).

We now propose to do this via the Intermediate Data Facility. This facility would only be for
pre operation activities e.g. serving data and training operations staff.
Commissioning actives would continue at NCSA and Chile.

Here we discuss DP0 in more detail.

\subsection {Interface}
More ambitiously than  \citeds{LSO-011} we would like to allow access to DP0 via
the Science Platform.
\subsubsection{Images}
Images would be accessible via a read only Gen3 butler repo.
\subsubsection{Catalogs}
Catalogs would be served via the VO TAP interface preferably backed by Qserv.



\subsection {Dataset(s)} \label{sec:dataset}

For DP0 we will use existing catalogs and products, images should be in a Gen3 Object Store backed repository (with S3 interface).

From a usability perspective the best data set to use would be HSC PDR2. Though public use of
this should be cleared with the HSC colleagues. This data set is well understood and semi regularly processed with our stack.  It is real data which is potentially more interesting for exploration than simulated data. The current repos are Gen2 so would need conversion meaning some subset may have to be used.

The main simulated data set would be DESC DC2. We need permission from the DESC colleagues. This is a large dataset and therefore interesting. It is currently only in Gen2 butler. There is a conversion script, however it
runs serially and would take prohibitively long to convert to Gen3. We could use a subset which would be fine.  Putting DC2 catalogs in Qserv would be an excellent demonstration of its abilities.


\subsection{Risks and mitigation}

The biggest schedule risk is not getting an interim data facility in place in time.
This would delay the entire schedule and there is not much mitigation.

In the long run costs may be higher than expected in a cloud based IDF. This will be due to storage.
An mitigation to this would be to store data on our own systems (NCSA or Chile) and expose it through S3.
NCSA already have this in place and we should consider testing this for lesser used data sets.


\section{Planning and team(s) fro DP0} \label{sec:plan}


\subsection {Teams}

The Operations era org chart is shown in \figref{fiq:org}.


\begin{figure}
\includegraphics[width=0.6\textwidth]{images/Ops_Org_Chart}
\caption{ Organization of departments and teams  for operations of \RO \label{fig:org}}
\end{figure}

The main departments involved in DP0 are Data Production and System Performance. With in those departments various people will be involved from the underlying teams but in small numbers. It makes most sense to approach DP0 with a task force approach. This might best be seen as two teams:

\begin{itemize}
\item Data production - with a focus on middleware and execution (\secref{sec:dp});
\item System Performance - with a focus on quality assurance and community support (\secref{sec:sp}).
\end{itemize}

As we advance the teams grow and we will transition to the an organization as in \figref{fig:org}
with team leads for each team.

\subsection{DP Middleware and Execution}\label{sec:dp}
For DP0 on IDF Hsin-Fang Chiang would coordinate Data Production activities and be the point
of contact for the IDF provider.
There is preops effort (fractional FTE) available in Execution and Pipelines as well as Middleware teams.

{\color{red} Should we start to list names here?}

\subsection{SP Quality  and Community Support} \label{sec:sp}

{\bf Leanne .. }

\appendix
% Include all the relevant bib files.
% https://lsst-texmf.lsst.io/lsstdoc.html#bibliographies
\section{References} \label{sec:bib}
\bibliography{local,lsst,lsst-dm,refs_ads,refs,books}

% Make sure lsst-texmf/bin/generateAcronyms.py is in your path
\section{Acronyms} \label{sec:acronyms}
\addtocounter{table}{-1}
\begin{longtable}{p{0.145\textwidth}p{0.8\textwidth}}\hline
\textbf{Acronym} & \textbf{Description}  \\\hline

 &  \\\hline
AGN & active galactic nuclei \\\hline
BNL & Brookhaven National Laboratory \\\hline
BPS & Batch Production Service \\\hline
DAGMan & Directed Acyclic Graph Manager \\\hline
DAX & Data Access Services \\\hline
DC2 & Data Challenge 2 (DESC) \\\hline
DESC & Dark Energy Science Collaboration \\\hline
DMTN & DM Technical Note \\\hline
DOE & Department of Energy \\\hline
DP & Data Production \\\hline
DP0 & Data Preview 0 \\\hline
DR1 & Data Release 1 \\\hline
DRP & Data Release Production \\\hline
FTE & Full-Time Equivalent \\\hline
FY21 & Financial Year 21 \\\hline
HSC & Hyper Suprime-Cam \\\hline
IDF & Interim Data Facility \\\hline
IN2P3 & Institut National de Physique Nucléaire et de Physique des Particules \\\hline
LDM & LSST Data Management (Document Handle) \\\hline
LSST & Legacy Survey of Space and Time (formerly Large Synoptic Survey Telescope) \\\hline
MOU & Memo Of Understanding \\\hline
NCSA & National Center for Supercomputing Applications \\\hline
OPS & Operations \\\hline
PCW & Project Community Workshop \\\hline
PDR2 & Public Data Release 2 (HSC) \\\hline
RTN & Rubin Technical Note \\\hline
S3 & (Amazon) Simple Storage Service  \\\hline
SP & Story Point \\\hline
TAP & Table Access Protocol \\\hline
\end{longtable}

% If you want glossary uncomment below -- comment out the two lines above
%\printglossaries





\end{document}
