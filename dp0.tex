\section{Data Preview 0}\label{sec:dp0}
The first ideas about initial releases of \RO data were presented in \citeds{LSO-011}.
There have since been delays in construction such that we have considered making Data Previews with
simulated data or other non \RO data (see \secref{datasets}).

We now propose to do this via the Intermediate Data Facility. This facility would only be for
pre operation activities e.g. serving data and training operations staff.
Commissioning actives would continue at NCSA and Chile.

Here we discuss DP0 in more detail.

\subsection {Interface}
More ambitiously than  \citeds{LSO-011} we would like to allow access to DP0 via
the Science Platform.
\subsubsection{Images}
Images would be accessible via a read only Gen3 butler repo.
\subsubsection{Catalogs}
Catalogs would be served via the VO TAP interface preferably backed by Qserv.



\subsection {Dataset(s)} \label{sec:dataset}

For DP0 we will use existing catalogs and products, images should be in a Gen3 S3 backed repository.

From a usability perspective the best data set to use would be HSC PDR2. Though public use of
this should be cleared with the HSC colleagues. This data set is well understood and semi regularly processed with our stack - it is therefore available in butler Gen3. It is real data which is potentially
more interesting for exploration than simulated data.

The main simulated data set would be DESC DC2. We need permission from the DESC colleagues. This is a large dataset and therefore interesting. It is currently only in Gen2 butler. There is a conversion script, however it
runs serially and would take prohibitively long to convert to Gen3. We could use a subset which would be fine.  Putting DC2 catalogs in Qserv would be an excellent demonstration of its abilities.


\subsection{Risks and mitigation}

The biggest schedule risk is not getting an interim data facility in place in time.
This would delay the entire schedule and there is not much mitigation.

In the long run costs may be higher than expected in a cloud based IDF. This will be due to storage.
An mitigation to this would be to store data on our own systems (NCSA or Chile) and expose it through S3.
NCSA already have this in place and we should consider testing this for lesser used data sets.

