\section{Data Preview 0}\label{sec:dp0}

In \citeds{LSO-011} we outlined a number of scenarios for early releases of \RO data.
The purpose of the these releases are not only to prepare the community for LSST data, but also to serve as an early integration test of existing elements of the Data Management systems and to familiarize the community with our access mechanisms.  

Two major new developments have occured since \citeds{LSO-011} was drafted:

\begin{itemize}
  
\item There have since been delays in construction such that we are now planning on making Data Previews with \RO simulated data or on-sky data from other observatories (see \secref{datasets}) which would still allow us to meet some of the goals of the early releases. 

\item We are planning on carrying these activities at the Intermediate Data Facility, which is is dedicated to Pre-Ops activities infrastructure needs such as serving data and training operations staff (commissioning actives will continue at NCSA and Chile).

\end{itemize}

In this document we outline notable elements of DP0, the first of these planned data previews, from the Data Management and Pre-Operations perspective. 


\subsection {Interface}
More ambitiously than  \citeds{LSO-011} we would like to allow access to DP0 via
the Science Platform.
\subsubsection{Images}
Images would be accessible via a read only Gen3 butler repo.
\subsubsection{Catalogs}
Catalogs would be served via the VO TAP interface preferably backed by Qserv.



\subsection {Dataset(s)} \label{sec:dataset}

For DP0 we will use existing catalogs and products, images should be in a Gen3 Object Store backed repository (with S3 interface).

From a usability perspective the best data set to use would be HSC PDR2. Though public use of
this should be cleared with the HSC colleagues. This data set is well understood and semi regularly processed with our stack.  It is real data which is potentially more interesting for exploration than simulated data. The current repos are Gen2 so would need conversion meaning some subset may have to be used.

The main simulated data set would be DESC DC2. We need permission from the DESC colleagues. This is a large dataset and therefore interesting. It is currently only in Gen2 butler. There is a conversion script, however it
runs serially and would take prohibitively long to convert to Gen3. We could use a subset which would be fine.  Putting DC2 catalogs in Qserv would be an excellent demonstration of its abilities.


\subsection{Risks and mitigation}

The biggest schedule risk is not getting an interim data facility in place in time.
This would delay the entire schedule and there is not much mitigation.

In the long run costs may be higher than expected in a cloud based IDF. This will be due to storage.
An mitigation to this would be to store data on our own systems (NCSA or Chile) and expose it through S3.
NCSA already have this in place and we should consider testing this for lesser used data sets.

